\clearpage
\onecolumn
\begin{landscape}
	\section{Anhang}
	\subsection{Approximation des Bode-Diagramms}\label{approx_bode}
	\renewcommand{\arraystretch}{1.5}
	\begin{longtable}{|p{5cm}|l|ll|ll|}
		\hline
		\textbf{Pole} & 
		\textbf{UTF} $H(s)$ &
		\multicolumn{2}{c}{\textbf{Amplitude} $|H(s)|$} & 
		\multicolumn{2}{|c|}{\textbf{Phase} $\angle(H(s))$}
		\\ \hline
		Keine, konstanter Faktor &
		$\alpha e^{j \beta}$ &
		\parbox[c][1cm]{1cm}{\includegraphics[width=1cm]{./images/bode-approx-konst.png}} &
		\, Konstant: $20 \log \alpha$ &
		\parbox[c][1cm]{1cm}{\includegraphics[width=1cm]{./images/bode-approx-konst.png}} &
		Konstant: $\beta$
		\\ \hline
		Pol im Ursprung &
		$\frac{\alpha}{s}$ &
		\parbox[c][1cm]{1cm}{\includegraphics[width=1cm]{./images/bode-approx-ampl-tp-ord1.png}} & 
		\begin{tabular}{l}
			Lineare Steigung: $-20 dB/Dek.$ \\
			$0dB$ bei $\omega = \alpha$
		\end{tabular} &
		\parbox[c][1cm]{1cm}{\includegraphics[width=1cm]{./images/bode-approx-konst.png}} & 
		Konstant: $-\frac{\pi}{2}$ 
		\\ \hline
		Nullstelle im Ursprung &
		$\alpha s$ &
		\parbox[c][1cm]{1cm}{\includegraphics[width=1cm]{./images/bode-approx-ampl-hp-ord1.png}} & 
		\begin{tabular}{l}
			Lineare Steigung: $+20 dB/Dek.$ \\
			$0dB$ bei $\omega = \frac{1}{\alpha}$
		\end{tabular} &
		\parbox[c][1cm]{1cm}{\includegraphics[width=1cm]{./images/bode-approx-konst.png}} &
		Konstant: $+\frac{\pi}{2}$
		\\ \hline	
		Reeller Pol (TP 1.Ord) &
		$\frac{1}{s + \alpha}$ &
		\parbox[c][1cm]{1cm}{\includegraphics[width=1cm]{./images/bode-approx-ampl-4.png}} &
		\begin{tabular}{ll}
			$\omega < \alpha$: & Konstant $-20 \log \alpha$  \\
			$\omega > \alpha$: & $-20dB/Dek.$
		\end{tabular} &
		\parbox[c][1cm]{1cm}{\includegraphics[width=1cm]{./images/bode-approx-phase-4.png}} &
		\begin{tabular}{ll}
			$\omega < \frac{\alpha}{10} $:	& Konstant $0$ \\
			$\omega > 10 \alpha$:		& Konstant $-\frac{\pi}{2}$
		\end{tabular}
		\\ \hline
		Reeller Pol (TP 1.Ord) &
		$\frac{\alpha}{s + \alpha}$ &
		\parbox[c][1cm]{1cm}{\includegraphics[width=1cm]{./images/bode-approx-ampl-4.png}} &
		\begin{tabular}{ll}
			$\omega < \alpha$: & Konstant $0dB$ \\
			$\omega > \alpha$: & $-20dB/Dek. \qquad (\omega_r = \alpha)$
		\end{tabular} &
		\parbox[c][1cm]{1cm}{\includegraphics[width=1cm]{./images/bode-approx-phase-4.png}}	& 
		\begin{tabular}{ll}
			$\omega < \frac{\alpha}{10}$: & Konstant $0$ \\
			$\omega > 10 \alpha$: & Konstant $-\frac{\pi}{2}$
		\end{tabular}
		\\ \hline
		Reelle Nullstelle (HP 1.Ord) &
		$s + \alpha$ & 
		\parbox[c][1cm]{1cm}{\includegraphics[width=1cm]{./images/bode-approx-ampl-5.png}} &
		\begin{tabular}{ll}
			$\omega < \alpha$: & Konstant $20 \log \alpha$ \\
			$\omega > \alpha$: & $+20dB/Dek.$
		\end{tabular} & 
		\parbox[c][1cm]{1cm}{\includegraphics[width=1cm]{./images/bode-approx-phase-5.png}}	&
		\begin{tabular}{ll}
			$\omega < \frac{\alpha}{10}$: & Konstant $0$ \\
			$\omega > 10 \alpha$: & Konstant $+\frac{\pi}{2}$
		\end{tabular}
		\\ \hline	
		Reelle Nullstelle (HP 1.Ord) &
		$\frac{s + \alpha}{\alpha}$ &
		\parbox[c][1cm]{1cm}{\includegraphics[width=1cm]{./images/bode-approx-ampl-5.png}} &
		\begin{tabular}{ll}
			$\omega < \alpha$: & Konstant $0dB$ \\
			$\omega > \alpha$: & $+20dB/Dek.$
		\end{tabular} &
		\parbox[c][1cm]{1cm}{\includegraphics[width=1cm]{./images/bode-approx-phase-5.png}} &
		\begin{tabular}{ll}
			$\omega < \frac{\alpha}{10}$: & Konstant $0$ \\
			$\omega > 10 \alpha$: & Konstant $+\frac{\pi}{2}$
		\end{tabular}
		\\ \hline
		Konjugiert-komplexe Pole \newline
		für $|q_p| > 1/2$ (TP 2.Ord) &
		$\frac{1}{s^2+s\frac{\omega_p}{q_p}+\omega_p^2}$ &
		\parbox[c][1cm]{1cm}{\includegraphics[width=1cm]{./images/bode-approx-ampl-6.png}} &
		\begin{tabular}{ll}
			$\omega < \omega_p$: 	& Konstant $-40 \log \omega_p$ \\
			$\omega > \omega_p$:	& $-40dB/Dek.$ \\
			Überhöhung: 			& $\frac{\omega_p}{2}$ bis $2\omega_p$ \\
			Maximum:				& $-40\log\omega_p + 20 \log q_p$ bei $\omega = \omega_p$			
		\end{tabular} &
		\parbox[c][1cm]{1cm}{\includegraphics[width=1cm]{./images/bode-approx-phase-6.png}} &
		\begin{tabular}{ll}
			$\omega < \frac{\omega_p}{10^{\frac{1}{2q_p}}}$:	& Konstant $0$ \\
			$\omega > \omega_p 10^{\frac{1}{2q_p}}$:			& Konstant $-\pi$ \\
			$\omega = \omega_p$:								& $-\frac{\pi}{2}$
		\end{tabular}
		\\ \hline
		Konjugiert-komplexe Pole \newline
		für $|q_p| > 1/2$ (TP 2.Ord)&
		$\frac{\omega_p^2}{s^2+s\frac{\omega_p}{q_p}+\omega_p^2}$ & 
		\parbox[c][1cm]{1cm}{\includegraphics[width=1cm]{./images/bode-approx-ampl-6.png}} &
		\begin{tabular}{ll}
			$\omega < \omega_p$:	& Konstant $0dB$ \\
			$\omega > \omega_p$:	& $-40dB/Dek.$ \\
			Überhöhung:				& $\frac{\omega_p}{2}$ bis $2 \omega_p$ \\
			Maximum:				& $20 \log q_p$ bei $\omega = \omega_p$
		\end{tabular} &
		\parbox[c][1cm]{1cm}{\includegraphics[width=1cm]{./images/bode-approx-phase-6.png}}	& 
		\begin{tabular}{ll}
			$\omega < \frac{\omega_p}{10^{\frac{1}{2q_p}}}$:	& Konstant $0$ \\
			$\omega > \omega_p 10^{\frac{1}{2q_p}}$:			& Konstant $-\pi$ \\
			$\omega = \omega_p$:								& $-\frac{\pi}{2}$
		\end{tabular}
		\\ \hline	
		Konjugiert-komplexe Nullstellen
		für $|q_z| > 1/2$ (HP 2.Ord)&
		$s^2+s\frac{\omega_z}{q_z}+\omega_z^2$ &
		\multicolumn{4}{l|}{
			Analog zu den Konjugiert-komplexen Polen jedoch gespiegelt an der $0dB$- / $0$-Grad-Linie.
		}
		\\
		&
		$\frac{s^2+s\frac{\omega_z}{q_z}+\omega_z^2}{\omega_z^2}$ &
		\multicolumn{4}{l|}{}
		\\ \hline
		\multicolumn{6}{|p{21cm}|}{
			Serieschaltung von Systemen erfolgt durch \textbf{Superposition} der einzelnen Bode-Diagramme 
			(Multiplikation von UTFs entspricht Addition im	dB-Bereich). $\alpha , \beta \in \mathbf{R}$. 
			Für $\omega_p$ und $q_p$ siehe Kapitel\ref{frequenzgang}
		}
		\\ \hline
	\end{longtable}
	\renewcommand{\arraystretch}{\arraystretchOriginal}
\end{landscape}
\clearpage

\subsection{Grundglieder}
Weitere Grundglieder auf \script{202}
\begin{longtable}{|c|c|l|}
	\specialrule{2pt}{0pt}{0pt}
	{\bf Typ} & {\it Symbol} & {\it Gleichung, DGL}\\
	& & {\it Sprungantwort}\\
	& & {\it Frequenzgang, Betrag und Argument}\\ \cline{2-3}
	& Strukturbild & {\it Nyquistdiagramm} -- {\it Bodediagramm}\\
	\specialrule{2pt}{0pt}{0pt}
	
	
	%P-Glied
	P & \parbox[c][2cm]{3cm}{\input{./tikz/PGlied}}
	&
	\begin{tabular}{lll}
		$y = Ku$ 		& 							& \\
		$u=1(t)$ 		& $y=K 1(t)$ 				& \\
		$G(j \omega)=K$	& $\left| G \right| = K$	& $arg(G)=0$ \\
	\end{tabular} 
	\\ \cline{2-3}
	& \parbox[c][2cm]{3cm}{\input{./tikz/PGliedStruktur}}
	& 
	\parbox[c]{3cm}{\input{./tikz/PGliedNyquist}} \quad
	\parbox[c]{6cm}{\input{./tikz/PGliedBode}}			 
	\\
	\specialrule{2pt}{0pt}{0pt}
	
	
	%I-Glied
	I & \parbox[c][2cm]{3cm}{\input{./tikz/IGlied}}
	&
	\begin{tabular}{lll}
		$\dot{y} = Ku$ 					
		& \multicolumn{2}{l}{$y = K \int\limits_{0}^{t}u(\tau)\;d\tau \qquad y(0) = 0 \qquad [K] = sec^{-1}$}										\\
		$u=1(t)$ 						& $y=K t$ 								& \\
		$G(j \omega)=\frac{K}{j\omega}$ & $\left| G \right| = \frac{K}{\omega}$ & $arg(G)=-\frac{\pi}{2}$ \\
	\end{tabular}
	\\ \cline{2-3}
	& \parbox[c][2cm]{3cm}{\input{./tikz/IGliedStruktur}}
	&
	\parbox[c]{3cm}{\input{./tikz/IGliedNyquist}}
	\parbox[c]{6cm}{\input{./tikz/IGliedBode}} 
	\\
	\specialrule{2pt}{0pt}{0pt}
	
	
	%D-Glied
	D & \parbox[c][2cm]{3cm}{\input{./tikz/DGlied}}
	&
	\begin{tabular}{lll}
		$y = K\dot{u}$					
		&	$[K] =sec$					& \\
		$u=1(t)$						& $y=K \delta (t)$						& \\
		$G(j \omega)=K j\omega$			& $\left| G \right| = K\omega$			& $arg(G)=\frac{\pi}{2}$
	\end{tabular}
	\\ \cline{2-3}
	& \parbox[c][2cm]{3cm}{\input{./tikz/DGliedStruktur}}			
	&
	\parbox[c]{3cm}{\input{./tikz/DGliedNyquist}}
	\parbox[c]{6cm}{\input{./tikz/DGliedBode}} 
	\\
	\specialrule{2pt}{0pt}{0pt}
		\newpage
\specialrule{2pt}{0pt}{0pt}
	
	%PT1_Glied
	$PT_1$ & \parbox[c][2cm]{3cm}{\input{./tikz/PT1Glied}}
	&
	\begin{tabular}{lll}
		$T\dot{y}+y=Ku$							& $y(0)=0$									& \\
		$u=1(t)$								& $y=K \left[ 1-e^{- \frac{t}{T}}\right]$	& \\
		$G(j \omega)= \frac{K}{1+j\omega T}$	& $\left| G \right| = \frac{K}{\sqrt{1+(\omega T)^2}}$ &
		$arg(G)=-\arctan(\omega T)$
	\end{tabular}
	\\ \cline{2-3}
	& \parbox[c][2cm]{3cm}{\input{./tikz/PT1GliedStruktur}}	
	&
	\parbox[c]{3.7cm}{\input{./tikz/PT1GliedNyquist}}
	\parbox[c]{6cm}{\input{./tikz/PT1GliedBode}}
	\\
	\specialrule{2pt}{0pt}{0pt}
	

	
	%PT2_Glied
	$PT_2$ &
	\begin{minipage}{3cm}
		\input{./tikz/PT2Glied}
	\end{minipage}
	&
	\begin{tabular}{lll}
		$T^2\ddot{y}+2\zeta T \dot{y}+y=Ku$ & $\ddot{y}+2\zeta\omega_n \dot{y}+\omega_n^2y=K\omega_n^2u$ & \\
		$y(0)=0$ & $\dot{y}(0)=0$ & $\omega_n=\frac{1}{T}$ \\
		\multicolumn{3}{l}{
			$y=K \left[1-\frac{1}{\sqrt{1-\zeta^2}}e^{-\zeta\omega_n t}\sin
			\left( \sqrt{1-\zeta^2} \omega_n t+arcos(\zeta) \right)\right]$
		} \\
		$G(j \omega)= \frac{K}{1+ 2 \zeta (j\omega) T  + (j \omega T)^2}$ & $\left| G \right| = \frac{K}{\sqrt{\left[1+(j\omega
				T)^2\right]^2+\left[2\zeta \omega T \right]^2}}$ & \\
		$\arg(G)=-\arctan  \frac{2\zeta \omega T}{1+(j\omega T)^2}$ & $0 \leq\omega T \leq 1$ & \\
		$\arg(G)=\arctan \frac{2\zeta \omega T}{1+(j \omega T)^2}-\pi$ & $1 \leq\omega T \leq \infty$ & \\
		
	\end{tabular}
	\\ \cline{2-3}
	& \parbox[c][2cm]{3cm}{\input{./tikz/PT2GliedStruktur}}
	& \begin{minipage}{4cm}
		\includegraphics[angle = {-0.3}, width=4cm]{./images/PT2_Nyq.jpg}
	\end{minipage}
	\begin{minipage}{8cm}
		\includegraphics[angle = {0.2}, width=7cm]{./images/PT2_Bode.jpg}
	\end{minipage} \rule[-5mm]{0mm}{35mm}
	\\
	\specialrule{2pt}{0pt}{0pt}
	
	
	%Tt_Glied
	$T_t$ &
	\parbox[c][2cm]{3cm}{\input{./tikz/TtGlied}}
	&
	\begin{tabular}{lll}
		$y=\begin{cases}
			0 & 0<t<T_t \\
			u(t-T_t) & t \geq T_t
		\end{cases}$ & & \\
		$u=1(t)$ & $y=1(t-T_t)$ & \\
		$G(j \omega)= e^{-j\omega T_t}$ & $\left| G \right| = 1$ & $arg(G)=-\omega T_t$
	\end{tabular}
	\\ \cline{2-3}
	& \parbox[c][2cm]{3cm}{\input{./tikz/TtGliedStruktur}}
	& 
	\parbox[c]{3cm}{\input{./tikz/TtGliedNyquist}}
	\parbox[c]{6cm}{\input{./tikz/TtGliedBode}}
	\\
	\specialrule{2pt}{0pt}{0pt}
	
	%DT1_Glied
	$DT_1$ &
	\parbox[c][2cm]{3cm}{\input{./tikz/DT1GliedStruktur}}
	&
	\begin{tabular}{lll}
		$G(j \omega)= G_D \cdot G_{PT_1} = j\omega T_V \dfrac{1}{1+j\omega T_C} = \dfrac{T_V}{T_C}\left(1- \dfrac{1}{1+ j\omega T_C} \right)$ \\
	\end{tabular}\\
	\specialrule{2pt}{0pt}{0pt}
\end{longtable}
\clearpage
\twocolumn